%!TEX output_directory = aux_dir
\documentclass[../SommerstjernerA4.tex]{subfiles}

\begin{document}
\section{Romfartøy}
Sommernettene er kanskje ikke gode for stjernekikking, men de kan være gode for å observere satellitter. Det å observere satellitter er ofte vanskelig fordi de kan verken observeres på dagen (solen er for lys), eller midt på natten (satellittene er i skyggen til jorden). Om sommeren er det varmt nok til å stå ute og vente på at satellittene titter frem på himmelen.

Helt nødvendig for å kunne se spesifikke satellitter er kalkulatorer for å regne ut når, og hvor, satellittene er til en hver tid. Det finnes mange gode apper til mobil med alarm og gyroskop som hjelper deg å finne rett sted på himmelen. Heavens-Above som jeg nevnte i kapittel \ref{sec:resources} er også bra.

\subsection{ISS}
ISS kjenner dere kanskje til fra planetvisningen. Den internasjonale romstasjon har hatt mennesker tilstede	kontinuerlig i snart 17 år.

Det mange ikke vet er at det er nokså enkelt å se ISS med det blotte øye. Anbefal publikum å sjekke ut kalkulatorer for når ISS er synlig.

Jeg pleier å sjekke før visning om det er noen gode observasjonsmuligheter, og videreformidler dette til publikum. I Norge er ISS synlig fra 14. jul til 5. aug.

Dersom man ser ISS er det ofte at man kan se den på nytt etter ca. 90 minutter på samme sted. Da har romstasjonen gått en hel runde rundt Jorden.

\subsection{Rakettoppskytinger}
Rakettoppskytninger er veldig fascinerende, og kan være svært lærerike. SpaceX viser oppskytningene sine live med kommentatorer som forklarer hva som skjer under oppskytningen. De lander ofte rakettene sine, og det kan være overraskende for de fleste at det bare tar åtte minutter fra oppskyting til landing.

Her er det opp til verten å videreformidle om det er noen interessante oppskytninger i nær fremtid. Hvis tidspunktet er korrekt og auditoriet ikke brukes kan det være en idé å sette å liveshowet på storskjermen. Vi gjorde noe liknende under en solformørkelse for noen år siden.

\subsection{Iridiumglimt}
Iridiumglimt (Iridium flares) er en av de kuleste tingene man kan se med det blotte øye. Iridiumsatellittene er kommunikasjonssatellitter med store, blanke speil. Når speilene er i akkurat korrekt posisjon kan de reflektere solen rett ned mot en observatør på Jorden.

Et iridiumglimt ser ut som en svak stjerne som beveger seg over himmelen før den plutselig eksploderer i et svært kraftig ``glimt'', og kan i mange tilfeller lyse sterkere enn Venus.

Observasjonene er veldig avhengig av posisjon, og bare du beveger deg noen få kilometer kan lysstyrken endre seg mye. De beste er de med megnitude $-5$ og lavere.\footnote{Lavere magnitude betyr høyere lysstyrke.}

Dessverre blir satellittene erstattet av en nyere modell som ikke vil lage iridiumglimt, så det haster å observere ett.
\end{document}