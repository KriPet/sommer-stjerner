%!TEX output_directory = aux_dir
\documentclass[../SommerstjernerA4.tex]{subfiles}

\begin{document}
\section{Diverse små historier}
Her har jeg plassert mindre temaer som ikke passet inn ellers.
\subsection{Arkturus}
Arkturus ($\alpha$ Boötis) er den sterkeste stjernen på den nordlige halvkule, og er en rød kjempe.

Den er også del av vårtriangelet (eller vårdiamanten), som kan indikeres i sammenheng med sommertriangelet.

\subsection{Meteorstorm}
Meteorstormen Perseidene er på sitt kraftigste den andre uken i August. Mer nøyaktige datoer kan du finne når vi nærmer oss.

Under Perseidene kan man observere ca. én meteor hvert minutt.

\subsection{Solformørkelse}
Den 21. august er det en total solformørkelse over USA. Den er ikke synlig i Europa, men kommer nok til å være i mediene rundt den tiden. Da kan det være en god ide å snakke om hvorfor vi har sol- og måneformørkelser. 

Tidligere har vi vist solformørkelse og Venus-passasje live i auditoriet, men i år skjer solformørkelsen om kvelden (ca. kl. 8), så be folk å se den hjemme.

I ZKPen kan man forklare solformørkelser ved å ``låse'' månefasen til full, og se hvordan månen overlapper solen. Dessverre får du ikke en skyggeeffekt.

\subsubsection*{Måneformørkelse}
ZKPen har en knapp som heter \zkp{GS} som står for Gegenschein. Gegenschein er lys fra støv i L$_2$ Lagrange-punktet, men sirkelen som ZKPen lager kan tenkes som Jordens skygge, og kan brukes til å vise hvordan månen beveger seg inn i Jordens skygge under en måneformørkelse.

Det er vanligvis en måneformørkelse en halv måned før eller etter en solformørkelse; og i år skjer det den 7. august kl. 20:20. Det er en veldig liten formørkelse, men er absolutt noe å anbefale besøkende å se.

\subsection{Zodiaken}
Sommervisninger er perfekte til å forklare hva Zodiaken er, og hvorfor man blir født inn i et stjernebilde. 

Jeg pleier å vise hvor solen er i Zodiaken og forklare at det stjernebildet solen ligger i forteller hvilke stjernebilde du er født i... men at det er lenge siden de datoene vi bruker var korrekte. Akkurat på samme måte som at nordstjernen beveger seg, roterer også Zodiaken i forhold til kalenderen. Det betyr at de som er født i løven/krepsen/tvillingene (spør publikum) er \emph{egentlig} født i krepsen/tvillingene/tyren; altså ett hakk ``bakover''.
\end{document}