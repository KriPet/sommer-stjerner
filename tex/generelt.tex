%!TEX output_directory = aux_dir
\documentclass[../SommerstjernerA4.tex]{subfiles}

\begin{document}
\section{Generelt}
Om sommernettene i Norge er det få stjerner å se. Det betyr likevel ikke at det ikke er mye å snakke om i planetariet. Dette skrivet forsøker å liste opp noen av de mer interessante temaene som er relevante i de korte sommernettene. Noen av emnene er spesifikke for sommeren 2017, mens andre er mer generelle.

\subsection{Fokus på det som er synlig}
Selv om det er fristende å snakke om de samme tingene du snakker om i vinterhalvåret er det ikke alt som passer like godt om sommeren. Andromedagalaksen er vanskelig nok å finne om vinteren, og er kanskje ikke like relevant om sommeren.

Forsøk i stedet å snakke om det man faktisk vil kunne se om man går ut på sommernettene. Husk at det er en del ting som faktisk er lettere om sommeren; det er for eksempel lettere å være ute takket være høyere temperaturer. Det er flere som går på overnattingsturer med telt, etc.

Sommertiden er også en perfekt unnskyldning til å snakke om hvorfor vi \emph{har} sommer, og hvorfor nettene er så korte.

Sammen med de få stjernene som faktisk er synlige, er det også en god ide å snakke om planetene som er synlige. Hvis du trenger noen ekstra minutter for å fylle ut en forestilling, er det mye informasjon om planetene.

Jeg syntes det er viktig å fortelle historier som publikum faktisk husker. Hvis du bare forteller om hvor gamle stjerner er, hvor langt borte de er, hvor raske stormene på Saturn er, etc., er det ingen som husker det etter fem minutter.

Hvis du absolutt har lyst til å snakke om ``de vanlige'' temaene, kan du spør ungene hvor de har vært/skal reise i ferien, som en unnskylding for å reise til sørlige trakter med lengre netter.

\subsection{Les deg opp hver uke}
Hver uke er det noe nytt som skjer på himmelen. Bruk ressursene i dette skrivet (Kapittel \ref{sec:resources}) til å se om det er synlige satellitter, meteorstormer, etc. Les deg opp om hva som skjer på ISS. Finn ut om noen av de synlige planetene har besøk av romfartøy.

Planetene trenger du ikke sjekke hver uke siden de beveger seg sakte, men hvis du ikke har hatt planetarievisning på en stund kan det være en god ide å sjekke hvor de er.

\subsection{Hvorfor har vi sommer?}
Et tema som kan være interessant for de litt eldre kan være å forklare hvorfor vi har sommer og vinter, og hvorfor nettene endrer lengde.

Den enkleste måten å gjøre dette på er å vise hvordan solen bare så vidt forsvinner under horisonten om sommeren, og sammenligne med en vinternatt, hvor solen bare så vidt står opp.

Skru på ekvator \zkp{EQU} og ekliptikken \zkp{EQL} og vis deretter med \zkp{ANN} knotten hvordan solen beveger seg langs ekliptikken. Ekvator holder seg fast hele tiden, mens solen beveger seg over og under ekvator.

Tips 1: Hvis du har lyst til å bevege solen raskt rundt ekliptikken kan det være lurt å skru av månen \zkp{BWD MO}. Hvis du vil at det skal gå enda raskere kan du prøve \zkp{ANN V VE}, som setter hastigheten til \zkp{ANN} lik hastigheten til Venus.

Tips 2: For enda mer avanserte publikum kan du forklare hva som skjer når du går til andre breddegrader; hvordan solen beveger seg på polene og ekvator.
\end{document}