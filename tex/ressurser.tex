%!TEX output_directory = aux_dir
\documentclass[../SommerstjernerA4.tex]{subfiles}

\begin{document}
\section{Ressurser}
\label{sec:resources}
I dette kapittelet vil jeg liste opp en del ressurser som kan være nyttige i planleggingen av stjernevisninger.

\subsection{Stellarium}
Stellarium (\url{http://www.stellarium.org/}) er et kjempebra gratis program til PC/Mac/Linux for å se stjernehimmelen. Det simulerer lysforurensning i og utenfor byer, og er god til å få en ide om hva som er synlig.

Du kan også søke etter objekter for å finne planeter eller satellitter. Det er også mulighet til å se hvor meteorstormer kommer fra.

\subsection{Heavens-Above}
Heavens-Above (\url{http://www.heavens-above.com/}) er en nettside for å regne ut hvilke satellitter som er synlige fra din posisjon på jorden. Veldig nyttig for å finne ut hvor og når du må se for å se ISS og iridiumglimt.

\subsection{Spaceflight Now}
Spaceflight now (\url{https://spaceflightnow.com/}) er en nettside om alt som har med raketter å gjøre. Her får du oversikt over hvilke raketter som skal skytes opp i nær fremtid.

\subsection{Yr.no}
Dersom du snakker om synlige stjerner er det greit om du sjekker skydekket. På yr.no sin detaljere time-for-time visning kan du se hvordan skydekket endres. Hvis det er overskyet kan du fortelle at stjernene og planetene ikke endrer seg mye, så det går fint an å gå ut en annen dag.

\end{document}